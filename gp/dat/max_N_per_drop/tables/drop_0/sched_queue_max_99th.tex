quelen-Q-W            & chi-15-5-tuple        & chi-15-* (global)     & chi-15-ipdst          & chi-15-ipdst/16       & chi-15-ipsrc,ipdst    & sj-12-5-tuple         & sj-12-* (global)      & sj-12-ipdst           & sj-12-ipdst/16        & sj-12-ipsrc,ipdst     & fb-web-5-tuple        & fb-web-* (global)     & fb-web-ipdst          & fb-web-ipdst/16       & fb-web-ipsrc,ipdst    & mawi-15-5-tuple       & mawi-15-* (global)    & mawi-15-ipdst         & mawi-15-ipdst/16      & mawi-15-ipsrc,ipdst  \\ \hline
10-1-1                & ? (?)                 & ? (?)                 & ? (?)                 & ? (?)                 & ? (?)                 & ? (?)                 & ? (?)                 & ? (?)                 & ? (?)                 & ? (?)                 & ? (?)                 & 1 (0s)                & ? (?)                 & ? (?)                 & ? (?)                 & ? (?)                 & ? (?)                 & ? (?)                 & ? (?)                 & ? (?)                \\ \hline
10-1-16               & 1 (0s)                & 1 (0s)                & 1 (0s)                & 1 (0s)                & 1 (0s)                & 1 (0s)                & 1 (0s)                & 1 (0s)                & 1 (0s)                & 1 (0s)                & ? (?)                 & ? (?)                 & ? (?)                 & ? (?)                 & ? (?)                 & 1 (0s)                & 1 (0s)                & 1 (0s)                & 1 (0s)                & 1 (0s)               \\ \hline
10-4-16               & 1 (0s)                & 1 (0s)                & 1 (0s)                & 1 (0s)                & 1 (0s)                & 1 (0s)                & 1 (0s)                & 1 (0s)                & 1 (0s)                & 1 (0s)                & ? (?)                 & ? (?)                 & ? (?)                 & ? (?)                 & ? (?)                 & 1 (0s)                & 1 (0s)                & 1 (0s)                & 1 (0s)                & 1 (0s)               \\ \hline
10-8-16               & 1 (0s)                & 1 (0s)                & 1 (0s)                & 1 (0s)                & 1 (0s)                & 1 (0s)                & 1 (0s)                & 1 (0s)                & 1 (0s)                & 1 (0s)                & ? (?)                 & ? (?)                 & ? (?)                 & ? (?)                 & ? (?)                 & 1 (0s)                & 1 (0s)                & 1 (0s)                & 1 (0s)                & 1 (0s)               \\ \hline
10-16-16              & 1 (0s)                & 1 (0s)                & 1 (0s)                & 1 (0s)                & 1 (0s)                & 1 (0s)                & 1 (0s)                & 1 (0s)                & 1 (0s)                & 1 (0s)                & ? (?)                 & ? (?)                 & ? (?)                 & ? (?)                 & ? (?)                 & 1 (0s)                & 1 (0s)                & 1 (0s)                & 1 (0s)                & 1 (0s)               \\ \hline
100-1-1               & ? (?)                 & ? (?)                 & ? (?)                 & ? (?)                 & ? (?)                 & ? (?)                 & ? (?)                 & ? (?)                 & ? (?)                 & ? (?)                 & ? (?)                 & 6 (10ns)              & ? (?)                 & ? (?)                 & ? (?)                 & ? (?)                 & ? (?)                 & ? (?)                 & ? (?)                 & ? (?)                \\ \hline
100-1-16              & 20 (174ns)            & 36 (282ns)            & 23 (190ns)            & 26 (230ns)            & 21 (170ns)            & 23 (49ns)             & 1 (0s)                & 1 (0s)                & 1 (0s)                & 25 (48ns)             & ? (?)                 & ? (?)                 & ? (?)                 & ? (?)                 & ? (?)                 & 12 (18ns)             & 1 (0s)                & 13 (20ns)             & 21 (35ns)             & 13 (18ns)            \\ \hline
100-4-16              & 17 (175ns)            & 36 (282ns)            & 17 (192ns)            & 35 (320ns)            & 16 (186ns)            & 32 (152ns)            & 1 (0s)                & 1 (0s)                & 1 (0s)                & 27 (116ns)            & ? (?)                 & ? (?)                 & ? (?)                 & ? (?)                 & ? (?)                 & 8 (12ns)              & 1 (0s)                & 9 (14ns)              & 14 (25ns)             & 8 (12ns)             \\ \hline
100-8-16              & 12 (137ns)            & 36 (282ns)            & 12 (144ns)            & 27 (259ns)            & 11 (143ns)            & 29 (133ns)            & 1 (0s)                & 1 (0s)                & 1 (0s)                & 26 (106ns)            & ? (?)                 & ? (?)                 & ? (?)                 & ? (?)                 & ? (?)                 & 8 (12ns)              & 1 (0s)                & 9 (14ns)              & 14 (25ns)             & 8 (12ns)             \\ \hline
100-16-16             & 10 (122ns)            & 36 (282ns)            & 9 (126ns)             & 23 (221ns)            & 9 (120ns)             & 29 (123ns)            & 1 (0s)                & 1 (0s)                & 1 (0s)                & 25 (100ns)            & ? (?)                 & ? (?)                 & ? (?)                 & ? (?)                 & ? (?)                 & 8 (11ns)              & 1 (0s)                & 9 (14ns)              & 14 (24ns)             & 7 (11ns)             \\ \hline
